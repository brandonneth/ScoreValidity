\documentclass{article}


\usepackage{comment}
\usepackage{graphicx}
\usepackage{csquotes}
\usepackage{balance}
\usepackage{setspace}

\usepackage{listings}
\usepackage{subcaption}
\usepackage{xcolor}

\lstset{ %
language=C++,                % choose the language of the code
basicstyle=\ttfamily\footnotesize,       % the size of the fonts that are used for the code
columns=fullflexible,
numbers=left,                   % where to put the line-numbers
numberstyle=\footnotesize,      % the size of the fonts that are used for the line-numbers
stepnumber=1,                   % the step between two line-numbers. If it is 1 each line will be numbered
numbersep=5pt,                  % how far the line-numbers are from the code
%backgroundcolor=\color{codeBG3},  % choose the background color. You must add \usepackage{color}
showspaces=false,               % show spaces adding particular underscores
showstringspaces=false,         % underline spaces within strings
showtabs=false,                 % show tabs within strings adding particular underscores
frame=single,           % adds a frame around the code
tabsize=2,          % sets default tabsize to 2 spaces
captionpos=b,           % sets the caption-position to bottom
breaklines=true,        % sets automatic line breaking
breakatwhitespace=false,    % sets if automatic breaks should only happen at whitespace
keywordstyle=\color{blue},       % keyword style
  %language=Octave,                 % the language of the code
  otherkeywords={SearchVar,MV,TSS,tileExpr,Search,tFunc...},           % if you want to add more keywords to the set
  numberstyle=\tiny\color{black}, % the style that is used for the line-numbers
  rulecolor=\color{black},
escapeinside={<@}{@>}
}


\newcommand{\todo}[1]{{\textcolor{red}{{\tt{TODO:}}\,\,#1 }}}
\newcommand{\an}[1]{{\textcolor{blue}{Author's Note: #1}}}
\newcommand{\ttt}[1]{{\texttt{#1}}}

\begin{document}

\section{Traversal Order}

In this section, we describe how RAJALC estimates the cost of a particular layout choice and how cost metrics are reused where possible.
When deciding between possible layout choices for a kernel, RAJALC determines the traversal order of each access in the kernel.
An access' traversal order describes the order in which a kernel traverses the dimensions of a view's data as they are laid out in memory. 
For example, the access \verb.A(i,j). to a row-major layout and the access \verb.B(j,i). to a column-major layout have the same traversal order because they both access memory with stride one.

\subsection{Deriving the Traversal Order}

Given an $n$-dimensional kernel that accesses a $d$-dimensional view, there are $n! * d!$ different possible combinations of kernel policy and data layout. 
Estimating the cost of all of these choices separately is prohibitively expensive, especially using any sort of dynamic benchmarking. 
However, the traversal order of an access compresses the kernel policy and data layout into a single feature that accurately predicts relative performance.

Listing~\ref{MatMulTraversalOrder} shows two implementations of a matrix multiplication using different kernel policies and involving views using different data layouts.
Deriving the traversal order for an access requires the extraction of three values: the policy order, the layout order, and the argument order. 
For the access to \verb.C. in \verb.knl1., the policy order is $(0,1,2)$, the layout order is $(0,1)$, and the argument order is $(0,1)$. 
In contrast, for the access to \verb.B. in \verb.knl2., the policy order is $(2,0,1)$, the layout order is $(1,0)$, and the argument order is $(2,1)$. 

With these values in hand, we can derive the traversal order, using the access to \verb.B. in \verb.knl2. as a running example. 
We start with the argument order $(2,1)$.
The first step is to normalize the policy order.
This step gives us the answer to the question: \enquote{What access has the same traversal order when the policy order is monotonically increasing (the normal policy order)?}
This is calculated using the following: \verb,[policy_order.indexof(arg) for arg in arg_order],. 
The index of $2$ in the policy is $0$, and the index of $1$ in the policy is $2$, so our intermediate value is $(0,2)$. 

The second step is to permute this intermediate value based on the data layout. 
We apply the layout permutation $(1,0)$ to our intermediate value $(0,2)$ to get our traversal order $(2,0)$. 
This traversal order tells us that in the access to \verb.B. in \verb.knl2., the dimension of \verb.B. with the largest stride is traversed by the iterator at nesting depth 2 (the innermost iterator) and the dimension of \verb.B. with the smallest stride is traversed by the iterator at nesting depth 0 (the outermost iterator). 
Listing~\ref{TraversalOrderDerivation} shows the algorithm for deriving the traversal order.

\begin{figure}
\begin{lstlisting}[
  label={MatMulTraversalOrder}, 
  caption={Two kernels implementing matrix multiplication using different kernel policies.}]
View2D A(A_data, layout_01);
View2D B(B_data, layout_10);
View2D C(C_data, layout_01);

auto loop_body = [=](auto i0, auto i1, auto i2) {
  C(i0,i1) += A(i0,i2) * B(i2,i1);
}

using Policy_012 = KernelPolicy<
  statement::For<0, loop_exec,
    statement::For<1, loop_exec,
      statement::For<2, loop_exec,
        statement::Lambda<0>
      >
    >
  >
>;
using Policy_201 = KernelPolicy<
  statement::For<2, loop_exec,
    statement::For<0, loop_exec,
      statement::For<1, loop_exec,
        statement::Lambda<0>
      >
    >
  >
>;

auto knl1 = make_kernel<Policy_012>(bounds, loop_body);
auto knl2 = make_kernel<Policy_201>(bounds, loop_body);
\end{lstlisting}

\end{figure}


\begin{figure}
\begin{lstlisting}[
  label={TraversalOrderDerivation}, 
  caption={Algorithm for deriving the traversal order of an access}
]
function derive_traversal_order(policy, layout, arguments) {
  intermediate = [policy.indexof(argument) for argument in arguments];
  traversal = [intermediate[p] for p in layout];
  return traversal;
}
\end{lstlisting}

\end{figure}

\subsection{Traversal Order as a Performance Metric}

Our claim is that the traversal order of an access is an accurate predictive metric for the relative performance of a layout choice. 
We support this claim empirically using performance data from three microbenchmarks.
For each microbenchmark, we record the 5-run average execution time of every combination of kernel policy and data layout. 
These execution times are then grouped by traversal order.
If our claim is valid, then the execution times for each traversal order will cluster and the groups of execution times will not overlap.

Microbenchmark 1 is an access to a 3-dimensional view in a 3-dimensional loop. 
Microbenchmark 2 is an access to a 2-dimensional view in a 3-dimensional loop, as in a matrix multiplication.
Microbenchmark 3 is an access to a 3-dimensional view in a 4-dimensional loop.



\begin{figure}
  \includegraphics{benchmark1_boxplot.pdf}
  \label{TraversalBenchmark1}
  \caption{Execution times for 3-dimensional loop accessing 3-dimensional view, grouped by traversal order.}
\end{figure}
Figure~\ref{TraversalBenchmark1} shows the execution times for the different traversal orders for microbenchmark 1. 
With the exception of traversal order $(2,1,0)$, the traversal orders show good clustering.
\todo{explanation for why the $(2,1,0)$ ones don't cluster as much}
Also, the six possible traversal orders show good differentiation, with the exception of $(0,1,2)$ and $(1,0,2)$. 
This is likely because the bulk of the performance improvement comes from the innermost loop in a nest traversing the stride 1 dimension of the data.



\begin{figure}
  \includegraphics{benchmark2_boxplot.pdf}
  \label{TraversalBenchmark2}
  \caption{Execution times for 3-dimensional loop accessing 2-dimensional view, grouped by traversal order.}
\end{figure}
Figure~\ref{TraversalBenchmark2} shows the execution times for the different traversal orders for microbenchmark 2. 
While the traversal orders for this microbenchmark are all highly clustered, they are less differentiated from one another. 
For example, we see that the orders $(0,2)$ and $(1,2)$ have similar performance, as do $(2,0)$ and $(2,1)$. 
The similarity in performance is again attributable to the influence of the position of the innermost loop iterator. 

\begin{figure}
  \includegraphics{benchmark3_boxplot.pdf}
  \label{TraversalBenchmark3}
  \caption{Execution times for 4-dimensional loop accessing 3-dimensional view, grouped by traversal order.}
\end{figure}

Figure~\ref{TraversalBenchmark3} shows the execution times for the different traversal orders for microbenchmark 3. 
A similar pattern as the previous microbenchmarks emerges here: grouping based on the position of the innermost iterator. 
\todo{describe.}
\end{document}